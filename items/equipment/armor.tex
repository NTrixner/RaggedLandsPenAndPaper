\chapter{Armor and Clothing}\label{ch:armor}
\section{Description}\label{armorDescription}
Every character can wear an armor, a helmet, a set of gloves, a set of boots, two rings and one amulet or other type of necklace.
Also, a character can wear one belt, which can house items that need to be accessed easily.\\
Armor is ordered into 4 weight categories: Cloth, Light, Medium and Heavy.
A character wearing armor gains the resistances against cutting, piercing and blunt damage specified in the "Res" column.
Also, any armor can have multiple attributes, which are described below.\\
While wearing armor that a character is proficient in using, they may add their Level to their Dodge value.
If a character is not trained in that type of armor, they are unable to do so.
\subsection{Armor quality}\label{subsec:armorQuality}
Manufactured armor can have different qualities.
For example, an iron plate armor might be rusted or it might be masterfully crafted.
The quality of cloth, or light armor adds a bonus to evade rolls - this modifier ranges from --3 (broken) to +3 (high mastery craftsmanship).
The quality of medium and heavy armor adds a bonus to block rolls - this modifier ranges from --3 (broken) to +3 (high mastery craftsmanship).
The quality of an armor changes its base price:\\

\begin{itemize}
    \item -3: Base Price * 1/4
    \item -2: Base Price * 1/3
    \item -1: Base Price * 1/2
    \item 0: Base Price
    \item +1: Base Price * 1.5
    \item +2: Base Price * 2
    \item +3: Base Price * 3
\end{itemize}

\section{Armor List}\label{armorList}
\subsection{Clothing}\label{subsec:clothing}

Clothing in itself does not count as armor, but it does grant Armor Reduction.

While wearing clothing and robes, a character counts as "unarmored", meaning that they can add their level to rolls for defensive actions.

\rowcolors{2}{lightgray}{white}
\begin{longtable}{p{3cm} | p{1.5cm} | p{5cm} | p{1cm} | p{1.5cm}}
	Name & Res &   Attributes & Size & Price\\ \hline
	Linen Jacket & 1c & Under & M & 40G\\
	
	Fleece Jacket & 1c & Warming 1 & M & 60G\\
	
	Spider Silk Jacket & 1c & Warming 1, Cooling 1, Enchantable 2, Under & M & 600G\\
	
	Mage Silk Jacket & 1c & Warming 2, Cooling 2, Magic Defense 1, Under & M & 1200G\\
	
	Linen Robes & 1c & Over & M & 85G\\
	
	Fleece Robes & 1c & Warming 1, Over & M & 70G\\

	Spider Silk Robes & 1c & Warming 1, Cooling 1, Enchantable 2 & M & 700G\\
	
	Mage Silk Robes & 1c & Warming 2, Cooling 2, Magic Defense 1 & M & 1200G\\
\end{longtable}

\subsection{Light Armor}\label{subsec:lightArmor}
Light armor is flexible and allows agile users to move around freely.
It can usually be donned and doffed in a matter of one or two minutes.
Leather, fur and very light, but also very expensive metals or metal chains are used to create light armor.
While wearing light armor, a character's Agility can not exceed 10.

\rowcolors{2}{lightgray}{white}
\begin{longtable}{p{3.5cm} | p{1.5cm} | p{5cm} | p{1cm} | p{1.25cm}}
	Name & Res &  Attributes & Size & Price\\ \hline
	Fur Jacket & 1b, 1c, 1p & Warming 2, Under, Pockets(2S) & M & 100G\\

	Leather Jacket & 2b, 2c, 2p & Warming 1, Cooling 1, Under, Pockets(4S)  & M & 250G\\

	Linen Gambeson & 3b, 2c, 1p & Warming 1, Under, Pockets(3S) & L & 180G\\

	Fleece Gambeson & 4b, 3c, 2p & Warming 1, Under, Pockets(3S) & L & 250G\\

	Bronze Chain Shirt & 2b, 5c, 2p & Over, Under, Pockets (2S) & M & 700G\\

	Iron Chain Shirt & 3b, 7c, 3p & Over, Under, Pockets(2S) & M & 1,200G\\
\end{longtable}

\onecolumn
\subsection{Medium Armor}
Medium Armor is a great balance between the strong defenses of heavy armor and the agility of light armor. It is often made of cloth with strips of metal sewn in, or with leather pads. It can also consist of overlapping slabs of metal or leather, giving the armor scales. It takes 5 Minutes to don or remove medium armor. Medium armor reduces the Agility of the wearer to a maximum value of 8.
\rowcolors{2}{lightgray}{white}
\begin{longtable}{p{3.5cm} | p{1.5cm} | p{5cm} | p{1cm} | p{1.25cm}}
	Name & Res & Attributes & Size & Price\\ \hline
	
	Fur Gambeson & 2c, 1p, 3b & Warming 2, Under & L & 225G\\
	
	Leather Gambeson & 4c, 2p, 5b & Isolating 1, Under & L & 450G\\
	
	Boiled Leather Gambeson & 4c, 2p, 5b & Under & L & 750G\\
	
	Reptile Leather Gambeson & 6c, 3p, 6b & Isolating 2, Under & L & 3,000G\\
	
	Salamander Leather Gambeson & 6c, 3p, 6b & Cooling 2, Warming 4, Under & L & 7,500G\\
	
	Lesser Dragon Leather Gambeson & 6c, 3p, 6b & Isolating 3, Elemental Defense 1, Under & L & 15,000G\\
	
	Dragon Leather Gambeson & 8c, 4p, 8b & Isolating 4, Elemental Defense 2, Enchantable 2, Under & L & 37,500G\\
	
	Wood Brigandine & 3c, 3p, 3b & Flammable, Living 1 & L & 200G\\
	
	Bone Brigandine & 4c, 4p, 4b & Insulating 2, Horrid, Enchantable 2 & L & 400G\\
	
	Iron Brigandine & 7c, 7p, 7b & - & L & 1,400G\\
	
	Steel Brigandine & 7c, 7p, 7b & - & L & 2,400G\\
	
	Mithral Brigandine & 9c, 9p, 9b & Cooling 1, Leightweight & L & 16,000G\\
	
	Tenthril Brigandine & 9c, 9p, 9b & Isolating 2, Indestructable 2, Enchanting 3 & L & 36,000G\\
	
	Wood Scale Mail & 5c, 3p, 3b & Flammable, Living 1 & L & 220G\\
	
	Bone Scale Mail & 6c, 3p, 3b & Insulating 2, Horrid, Enchantable 2 & L & 440G\\
	
	Iron Scale Mail & 9c, 5p, 5b & - & L & 1,540G\\
	
	Steel Scale Mail & 9c, 5p, 5b & - & L & 2,640G\\
	
	Mithral Scale Mail & 11c, 7p, 7b & Cooling 1, Leightweight & L & 17,600G\\
	
	Tenthril Scale Mail & 11c, 7p, 7b & Isolating 2, Indestructible 2, Enchanting 3 & L & 39,600G\\
\end{longtable}
\twocolumn

\subsection{Heavy Armor}\label{subsec:heavyArmor}
Heavy Armor is slow and clunky, but also adds the biggest bonus to armor of any type.
It requires at least 10 minutes to don and doff such a piece of armor, and while wearing Heavy Armor, the wearer's Agility can not exceed a value of 6.
\rowcolors{2}{lightgray}{white}
\begin{longtable}{p{3cm} | p{1.5cm} | p{5cm} | p{1cm} | p{1.5cm}}
	Name & Res &  Attributes & Size & Price\\ \hline
	Bronze Half Plate & 8c, 8p, 5b & Over, Pockets(2S) & L & 350G\\
	
	Iron Half Plate  & 11c, 11p, 7b & Over, Pockets(2S) & L & 800G\\

	Steel Half Plate  & 11c, 11p, 7b & Over, Pockets(2S) & L & 2,500G\\

	Bronze Full Plate & 9c, 9p, 7b & Over, Pockets(2S) & L & 750G\\

	Iron Full Plate  & 12c, 12p, 10b & Over, Pockets(2S) & XL & 1,500G\\

	Steel Full Plate  & 12c, 12p, 10b & Over, Pockets(2S) & XL & 3,200G\\
\end{longtable}

\section{Armor Attributes}\label{sec:armorAttributes}

\subsection{Warming}\label{armor:warming}
An Armor with this attribute helps against the cold.
It grants resistance to cold by the value specified.

\subsection{Cooling}\label{armor:cooling}
An Armor with this attribute helps against the heat.
It grants resistance to heat by the value specified.

\subsection{Shielding}\label{armor:shielding}
An Armor with this attribute grants damage resistance to Radiation damage by the amount specified.

\subsection{Insulating}\label{armor:insulating}
An Armor with this attribute grants damage resistance to Electrical damage by the amount specified.

\subsection{Living}\label{armor:living}
An Armor with this attribute grants damage resistance to Necrotic damage by the amount specified.

\subsection{Flammable}\label{armor:flammable}
An Armor with this attribute easily catches fire.
Whenever a character wearing it would take heat damage, the armor starts to burn.\\
Burning Armor can be put out by anyone who is able to touch it.
Doing so takes 4 AP.\\
Burning Armor is also extinguished when the wearer would take cold damage, or when it is doused in water.\\
If burning armor isn't put out by some other effect, it burns for 6 AP if it's cloth or light armor, 10 AP if it is medium or 14 AP if it is heavy.\\
While wearing a burning armor, the wearer takes 1 heat damage per AP that passes until it is put out.
Any armor that burned for at least 1 AP becomes damaged afterwards, losing all of its properties.

\subsection{Horrid}\label{armor:horrid}
An Armor with the horrid attribute has dark energies associated with it.
When the wearer of such armor takes necrose or psychic damage, they take an additional 3 damage from that damage source.

\subsection{Enchantable}\label{armor:enchantable}
An Armor that has this attribute can be enchanted a different amount than just once.
It can instead hold an amount of Enchantments by the number specified.
This means that any armor that doesn't have the "Enchantable" attribute specified has the "Enchantable 1".

\subsection{Lightweight}\label{armor:lightweight}
An Armor with the Leightweight attribute reduces the wearer's Agility as if it were one Weight class lower.
Additionally, if the wearer of said armor is more proficient in the lower weight class than the armor's actual weight class, he can benefit from the lower proficiency as if the armor was one weight class lower.

\subsection{Magic Defense}\label{armor:magicDefense}
An Armor with this attribute grants additional damage resistance to damage from spells, by the amount specified.

\subsection{Elemental Defense}\label{armor:elementalDefense}
An Armor with this attribute grants damage resistance to elemental damage by the amount specified, if no other aspect of the armor already grants larger resistance to that element.

\subsection{Indestructible}\label{armor:indestructible}
An Armor with this attribute grants damage resistance to Corrosion and Nihilation damage by the amount specified.

\subsection{Over and Under}\label{armor:under}
An armor with the "under" attribute can be worn under any "over" armor.\\
An armor with the "over" attribute can be worn over any "under" armor.\\
When both armors grant resistances to a damage type, do not add them together.
Instead, the larger respective resistance count.\\
Wearing two pieces of armor reduces a character's Agility by 2, to a minimum of 1.

\subsection{Hats and Helmets}\label{subsec:helmets}

\subsubsection{Knight's Helmet}\label{item:knightHelment}
Size: M\\
Price: 100G\\
The knight's helmet is a full metal helmet with a visor.
It reduces the wearer's perception by 2, to a minimum of 1.\\
If the wearer of this helmet is subject to a critical hit from a weapon attack, the attacker must roll a d6.
On a 1, the attack does not deal additional damage from being a critical hit.\\

\subsubsection{Mages' Hat}\label{item:mageHat}
Size: M\\
Prices: 150G\\
The mages' hat is a signifier of a studied magister of magic.
It is pointy and made of magical silk.\\
While wearing a Mages' Hat, the wearer can remember 4 additional spells.\\

\subsubsection{Leather Helmet}\label{item:leatherHelmet}
Size: M\\
Price: 20G\\
The Leather Helmet is a simple cap made of leather, with straps on the sides.\\
If the wearer of this helmet is subject to a critical hit from a weapon attack, the attacker must roll a d12.
On a 1, the attack does not deal additional damage from being a critical hit.\\

\subsubsection{Hood}\label{item:hood}
Size: M\\
Price: 5G\\
This hood protects the wearer's head of wind and rain and makes it more difficult to see their face from the side.\\

\subsubsection{Skullcap}\label{item:skullCap}
Size: M\\
Price: 50G\\
This metal cap hugs the wearer's skull in order to protect it.\\
If the wearer of this helmet is subject to a critical hit from a weapon attack, the attacker must roll a d8.
On a 1, the attack does not deal additional damage from being a critical hit.\\

\subsubsection{Faceless Mask}\label{item:facelessMask}
Size: M\\
Price: 70G\\
This special mask is completely featureless.
It blocks the user's sight, making them effectively blind.
However, it also increases their concentration.
While wearing a faceless mask, the mana costs for spell upkeep is halved.\\


\subsection{Gloves and Gauntlets}

\subsubsection{Leather Gloves}
Size: S\\
Price: 5G\\
These Leather Gloves protect the hands.

\subsubsection{Gauntlets}
Size: S\\
Price: 50G\\
These Gauntlets protect the hands.
When the wearer is targeted disarming attempt, they can add 5 to their opposing weapon attack roll.\\
Also, the wearer's unarmed attacks deal 1d8 damage instead of 1d4.

\subsubsection{Silk Gloves}
Size: S\\
Price: 15G\\
These gloves protect against the weather and help the wearer with delicate work.
When doing something filigree that requires a Dexterity check, the wearer can add 2 to that check.

\subsection{Boots and Shoes}

\subsubsection{Cloth boots}

\subsubsection{Leather boots}

\subsubsection{Sabaton}

\subsubsection{Boots of Longstride}

\subsection{Rings}

\subsubsection{Ring of Protection}

\subsubsection{Fox Ring}

\subsubsection{Ring of Health}

\subsubsection{Ring of Might}

\subsubsection{Ring of the Mage}

\subsubsection{Ring of Stars}

\subsection{Necklaces and Amulets}\label{subsec:amulets}

\subsubsection{Amulet of the Lord}\label{item:amuletOfLord}
Size: S\\
Price: 1000G\\
This amulet increases the wearer's Empathy by 1.

\subsubsection{Amulet of Vitality}\label{item:amuletOfVitality}
Size: S\\
Price: 1000G\\
This amulet increases the wearer's Vitality by 1.

\subsubsection{Amulet of Fatigue}\label{item:amuletOfFatigue}
Size: S\\
Price: 2500G\\
This amulet increases the wearer's Stamina by 2 for every rank of ~\nameref{perk:increasestamina} that they have.

\subsubsection{Amulet of Light}\label{item:amuletOfLight}
Size: S\\
Price: 1500G\\
This amulet allows the user to activate it for 2 AP by speaking its command phrase.
When doing so, the amulet starts to glow.
It sheds bright light in a radius of~\passus{6} and dim light beyond that for~\passus{6}.
Speaking its command phrase again ends the glowing.

\subsubsection{Amulet of Feathers}\label{item:amuletOfFeathers}
Size: S\\
Price: 1500G\\
This amulet allows the user to carry twice as much as they would be able to otherwise.

\subsubsection{Amulet of Poison Protection}\label{item:amuletOfAntidote}
Size: S\\
Price: 1500G\\
This Amulet allows you to roll twice whenever you try to resist poison effects, taking the higher result.
If a poison would damage you, you only take half damage from it.
This damage reduction is applied after the static reduction.

\subsection{Belts}

\subsubsection{Girdle}
Weight: -\\
Price: -\\

\subsubsection{Potion Belt, 3 Slots}
Weight: -\\
Price: -\\

\subsubsection{Potion Belt, 5 Slots}
Weight: -\\
Price: -\\