\chapter*{Runes}\label{ch:runes}

Runes are a written symbol that holds the ability to shape magical energy to fit a certain element or create certain manifestations.\\
A magic-user can wield runes by etching them into rocks, putting them into writing, or creating tattoos on their own bodies.
They can then channel Mana through the rune and create magical effects from them.\\
Each Rune has 3 Levels of understanding that can be mastered, and allows its user to create and manipulate the Rune's Element in some way when just wielding it.\\
If a rune's effect requires an attack, this is automatically treated as a magical weapon attack, and requires the rune to be either wielded as a rune stone, or be present on the wearer's body in some way.
Additionally, Spells are comprised of strange and arcane formulae, which are represented in written Form by assembling Runes.
Therefore, in order to learn and master a Spell, a Mage has to also understand the underlying Runes.
This is represented by the respective Requirements of each Spell in the Spells Section.

\section{Circles}\label{sec:arcaneCircles}
There are 14 runes in the Ragged Lands, and they are organized within circles.
These circles contain multiple similar, but opposed elements .
The two exceptions are the Arcana Rune, which produces a form of enhancement or type-less base energy, as well as the Nil Rune, which produces elemental nothingness, since this Element is opposed to all other elements except the pure and unchanged Arcana.
The Arcana Rune is part of its own circle, while the Nil Rune is part of all 4 other Circles.\\

Learning and understanding multiple Runes in the same Circle is difficult thanks to the opposing elements that they hold.
A Spellcaster has to wrap their Head around Elemental concepts that are directly opposed to each other.\\

As a result, when calculating the CP Costs of learning a new Rune, the costs of the new Rune Level are based on the total amount of Rune Levels already gained for this Circle.\\

This means that the costs for learning the next level of the Nil rune equal the costs for learning a rune in the circle where the caster has learned the least amount of runes.\\

However, it also means that learning a new level of the Nil rune increases the cost of learning runes in all 4 Circles.\\

The Circles and their respective Runes are:

\rowcolors{2}{lightgray}{white}
\begin{tabular}{c | c | c | c}
	\multicolumn{4}{c}{Circles and Runes}\\
	\multicolumn{4}{c}{Arcana} \\
	\multicolumn{4}{c}{Arcana} \\
	\hline
	\hline
	\multicolumn{4}{c}{Phota}\\
	Light & Dark & Shadow & Nil\\
	\hline
	\hline
	\multicolumn{4}{c}{Aggregaria}\\
	Water & Air & Earth & Nil\\
	\hline
	\hline
	\multicolumn{4}{c}{Motua}\\
	Heat & Lightning & Cold & Nil\\
	\hline
	\hline
	\multicolumn{4}{c}{Anima}\\
	Life & Death & Undeath & Nil\\
\end{tabular}
\\
\\
And the CP and Gold Costs for increasing a rune are as follows:
\\
\\
\rowcolors{2}{lightgray}{white}
\begin{tabular}{c | c | c || c | c | c}
	Level & CP & Gold & Level & CP & Gold \\ \hline
	I & 50 & 50 & VII & 1,400 & 1,400 \\
	II & 150 & 150 & VIII & 1,800 & 1,800 \\
	III & 300 & 300 & IX & 2,250 & 2,250 \\
	IV & 500 & 500 & X & 2,750 & 2,750 \\
	V & 750 & 750 & XI & 3,300 & 3,300 \\
	VI & 1,050 & 1,050 & XII & 3,900 & 3,900 \\
\end{tabular}

	\twocolumn
\section{Arcana Rune}\label{rune:arcana}
\textbf{Cost:} 50 CP\\
\textbf{Requirements:} -\\
\textbf{Rune, Weapon, Active, Repeatable, Source (50 Gold)}\\
You have mastered the Arcana Rune. It is associated with the basic flow of magic and manifests into an elemental force that is neutral of every element. While Arcana seems to be a very simple Rune, especially for new Apprentices, it has many strong powers and enjoys synergies with other Elements, allowing it to be used in a multitude of Spells.\\
\subsection{Level 1 Abilities}

\subsubsection{Arcane Bolt}
For 4 AP and 3 Mana, you create a bolt of magic out of thin air and hurl it towards an enemy.
Make a ranged magic weapon attack.
The bolt deals 2d4 piercing damage.

\subsubsection{Feel Arcane Energy}
Thanks to your attunement with the arcane energies, you can feel the presence of magic.
Any form of the Arcana Rune slightly starts glowing when it touches some magical item, unless that item is shielded from such detection in some form.

\subsection{Level 2 Abilities}
\textbf{Cost:} 500 CP\\
\textbf{Source:} 350 Gold
\subsubsection{Enhanced Arcane Bolt}
When creating an Arcane Bolt you can now spend 6 Mana to create two bolts instead.
Each of these bolts can target different enemies, and each of the bolts requires its own attack roll.

\subsubsection{Arcane Blast}
You can create a blast of magic energy at a point in a range of \passus{9} around you.
This takes 6 AP and 15 Mana.
Any creature in a radius of \passus{3} around your chosen point takes 4d6 blunt damage and must make a Strength check, contested by your Intelligence Check.
If they fail, they are pushed \passus{1} away from the point of origin of the blast, and fall prone.

\subsubsection{Feel Lingering Magic}
Your attunement to the magical energies around you have become stronger.
Now your Arcana Rune also starts glowing in an area where magic has been cast recently.
For every point of Mana spent in an area, the lingering aura lasts for an additional 10 minutes.

\subsection{Level 3 Abilities}
\textbf{Cost:} 1,000 CP\\
\textbf{Source:} 650 Gold
\subsubsection{Arcane Bolt}
When creating an Arcane Bolt you can now spend 9 Mana to create three bolts instead.
Each of these bolts can target different enemies, and each of the bolts requires its own attack roll.

\subsubsection{Strong Arcane Blast}
You can now create a stronger arcane blast.
The range in which you can choose its point of origin has increased to \passus{12} and the radius of the blast itself has increased to \passus{5}.\\
Additionally, you can now expand 20 Mana instead to deal 10d6 blunt damage.

\subsubsection{Arcane Sight}
You now have the ability to see magical energies circulating around you.
In order to do so, you must close your eyes and spend 1 Mana per Action Point.
Magical Energies swirl around the Universe like a web or radiation, allowing you to discern shapes of creatures and objects.
This type of Sight detects invisible creatures and is nearly 360 degrees.
\onecolumn
	
	\section{Light Rune}\label{rune:light}
\textbf{Cost:} See Runes\\
\textbf{Requirements:} -\\
\textbf{Rune, Weapon, Active, Repeatable, Source (See Runes)}\\
You have mastered the Light Rune.
Light is the element of sight and vision.
It allows sight and brings truth into dark places.
Those that wander in the path of the light never have to watch their step, as they are always on the right path.

\subsection{Rank I Abilities}

\subsubsection{Orb of Light}
For 4 AP you can create a small Orb of Light that shines a bright light for a radius of \passus{4}.
Beyond that, the light fades into dim light for a radius of \passus{4}.
Beyond that, darkness still prevails.
The Orb is the size of your hand and you can control its position telepathically.
However, it can not move further away than \passus{10} from you.
You can decide the colour of the orb and change it at any time. \\
Keeping the Orb of Light up costs 1 Mana per 10 Minutes.


\subsection{Rank II Abilities}

\subsubsection{Spirit of Light}
Your Orb of Light has become a small manifestation of elemental light that follows you around.
It now shines bright light for \passus{6} and dim light for an additional \passus{6}.\\
Additionally, it can now carry coin-sized objects.
However, while carrying such objects it requires 1 Mana per AP.

\subsubsection{Light Burst}
You can now create a burst of light from the Light Rune you are wielding.
Doing so takes 2 AP and 15 Mana.
In a \passus{4}-long cone in front of you, a burst of bright light comes forth.
All creatures in that area must make a Vitality Check against your Intelligence Check.
If they fail, they become blinded for 1D4 + 2 AP.

\subsection{Rank III Abilities}

\subsubsection{Vestige of Light}
Your Spirit of Light can now engulf any creature of your choice within range, making it more difficult to hit it with an attack.
While engulfing a creature in this way, the Dodge of that creature is increased by your Intelligence, but only against attackers that rely on sight.
Doing this costs 3 Mana per AP.

\subsubsection{Daylight Burst}
Your Light Burst has become powerful enough to damage creatures.
Performing a Daylight Burst costs 30 Mana instead of 15, and the cone has a size of \passus{6}.
Any creature inside the cone takes 4D6 Hot damage in addition to having a chance of being blinded.
Undead take 8D6 Hot damage instead.
Vampires take 12D6 Hot damage instead.

\subsubsection{Kaleidoscope}
You can create a burst of colours around you, dazzling and confusing each creature in the area.
Any creature in a radius of \passus{6} around you must make a Perception Check contested by your Intelligence Check.
If a creature fails, they are affected by a random Condition for 2D6 + your Intelligence AP.\\

\rowcolors{2}{lightgray}{white}
\begin{tabular}{r | l || r | l}
	1DX & Condition & 1DX & Condition\\
	\hline
	1 & Entranced & 5 & Unconscious \\
	2 & Frightened & 6 & Enraged \\
	3 & Sleeping & 7 & Calmed\\
	4 & Paralyzed & 8 & Now your ally \\
\end{tabular}
	
	\section{Dark Rune}\label{rune:dark}
	
	\section{Shadow Rune}\label{rune:shadow}
\textbf{Cost:} See Runes\\
\textbf{Requirements:} -\\
\textbf{Rune, Weapon, Active, Repeatable, Source (See Runes)}\\
You have mastered the Shadow Rune.
% TODO Add description

\subsection{Rank I Abilities}

\subsubsection{Shadowform}
You cast for 3 AP and 5 Mana, and your shadow rises from the ground and envelops your body.
While you are in this form, any attack roll that would hit you has 50\% chance of hitting your shadow instead of you.
Once the shadow is hit, it disappears and becomes your normal shadow again.
	
	\section{Water Rune}\label{rune:water}
\textbf{Cost:} See Runes\\
\textbf{Requirements:} -\\
\textbf{Rune, Weapon, Active, Repeatable, Source (See Runes)}\\
You have mastered the Earth Rune.
% TODO Add description

\subsection{Level 1 Abilities}

\subsubsection{Aquamancy}
For 4 AP and 2 Mana, you can start moving water surrounding you.
Any liquid in a radius of~\passus{12} of you can be moved.
Doing so takes 1 AP per Minute, and you can move a mass equal to 5 Stone (30 kg, 65 lb) per Minute.
You can not use this ability to move the liquid inside creatures.
You can use this ability to create small waves in ponds, change weak currents, or separate liquids

\subsubsection{Acid Burst}
For 4 AP and 4 Mana you create a burst of acid emanating from you, and damaging creatures surrounding you.
Make an Intelligence check.
Any creature in a radius of~\passus{3} of you must make an Agility check against it.
If a creature fails, it takes 2d6 corrosive damage.
If it succeeds, it takes no damage instead.
	
	\section{Air Rune}\label{rune:air}
\textbf{Cost:} See Runes\\
\textbf{Requirements:} -\\
\textbf{Rune, Weapon, Active, Repeatable, Source (See Runes)}\\
You have mastered the Air Rune.
% TODO Add description

\subsection{Rank I Abilities}

\subsubsection{Wind Gust}
For 2 AP and 4 Mana, you can create a small gust of wind in a range of \passus{12}.
This gust is strong enough to blow out candles, but makes larger fires only flicker.
You can also use it to move very light, small objects, like pieces of paper or feathers at a speed of \passus{1} per AP.
For every AP that you keep up this Wind Gust, you must pay 1 Mana.

\subsubsection{Air Blade}
For 3 AP and 8 Mana you create a Blade of Wind that travels in a straight line from your position to a target of your choice and has a range of \passus{6}.
Make a ranged magic weapon attack against that target.
On a hit, the Air Blade deals 3d6+3 cutting damage.
	
	\section{Earth Rune}\label{rune:earth}
\textbf{Cost:} See Runes\\
\textbf{Requirements:} -\\
\textbf{Rune, Weapon, Active, Repeatable, Source (See Runes)}\\
You have mastered the Earth Rune.
% TODO Add description

\subsection{Rank I Abilities}

\subsubsection{Geomancy}
For 3 AP and 5 Mana, you can start moving the earth surrounding you.
Any mineral compound with loose texture (Mud, Dirt, Sand, Clay) in a radius of~\passus{4} of you can be moved.
Doing so takes 1 Mana per Minute, and you can move a mass equal to 1 Stone (6 kg, 13 lb) per Minute.

\subsubsection{Hurl Rock}
For 3 AP and 4 Mana you move a rock from the ground and hurl it at an enemy in a range of \passus{6}.
Make a ranged spell attack.
On a hit, the target takes 1d12 blunt damage.
Additionally, if the target took damage from the rock, it is moved away from you by \passus{1} per point of damage it actually took.
	
	\section{Heat Rune}\label{rune:heat}
\textbf{Cost:} See Runes\\
\textbf{Requirements:} -\\
\textbf{Rune, Weapon, Active, Repeatable, Source (See Runes)}\\
You have mastered the Heat Rune.
% TODO Add description

\subsection{Rank I Abilities}

\subsubsection{Flame Hand}
For 3 AP and 4 Mana, you create a small flame in your hands.
This flame costs 1 Mana per 10 Minutes to keep burning, and sheds bright light in a radius of \passus{2} and dim light for an additional \passus{2} beyond that.
While holding the flame, flammable things that you touch with it start to ignite.
For 2 AP, you can hurl the flame.
It travels for a distance of up to \passus{12}.
If you hurl the flame at a creature, make a ranged spell weapon attack.
On a hit, the flame deals 2d8+2 heat damage.
The flame in your hand ends if you hurl it.
You can dismiss the flame in your hand at any point.
	
	\section{Lightning Rune}\label{rune:lightning}
\textbf{Cost:} See Runes\\
\textbf{Requirements:} -\\
\textbf{Rune, Weapon, Active, Repeatable, Source (See Runes)}\\
You have mastered the Lightning Rune.
% TODO Add description

\subsection{Rank I Abilities}

\subsubsection{Lightning Bolt}
For 3 AP and 4 Mana, you create a lightning bolt from your rune that targets a creature in a range of~\passus{12}.
Make a ranged spell attack.
On a hit, the lightning bolt deals 2d6+2 lightning damage, and 2 AP damage.
	
	\twocolumn
\section{Cold Rune}\label{rune:cold}
\textbf{Cost:} See Runes\\
\textbf{Requirements:} -\\
\textbf{Rune, Weapon, Active, Repeatable, Source (See Runes)}\\
You have mastered the Cold Rune.
% TODO Add description

\subsection{Level 1 Abilities}

\subsubsection{Ice Touch}
For 2 AP and 4 Mana you lower the temperature of your hand to a freezing point.
Keeping this effect up takes 1 Mana per AP.
When you touch an object or a creature, you can decide if you want to lower their temperature gradually or aggressively.
If you lower someone's or something's temperature gradually, you can do so at a rate of about 1 degree celsius or 1 Kelvin per 5 AP.
If you do so aggressively, make a melee spell attack against that target.
On a hit, the Ice Touch deals 2d4 cold damage and 2d4 Stamina damage.
	
	\section{Life Rune}\label{rune:life}
\textbf{Cost:} See Runes\\
\textbf{Requirements:} -\\
\textbf{Rune, Weapon, Active, Repeatable, Source (See Runes)}\\
You have mastered the Life Rune.
% TODO Add description

\subsection{Rank I Abilities}

\subsubsection{Curative Touch}
For 2 AP and 8 Mana, your rune starts to faintly glow with life energy.
Roll 4d4.
The rune gains a total healing power equal to the result of this roll.
When you use 3 AP to touch a creature with this rune afterwards, you can spend any amount of those points to heal the creature by this amount.
If a creature is undead or not willing, you have to make a melee spell attack instead.
An undead is damaged by the healing of this rune instead.
	
	\section{Death Rune}\label{rune:death}
	
	\section{Undeath Rune}\label{rune:undeath}
	
	\section{Nil Rune}\label{rune:nil}
\textbf{Cost:} See Runes\\
\textbf{Requirements:} -\\
\textbf{Rune, Weapon, Active, Repeatable, Source (See Runes)}\\
You have mastered the Nil Rune.
% TODO Add description

\subsection{Rank I Abilities}

\subsubsection{Antimagic}
When you are targeted by a spell or rune attack that would deal damage to you, you can use your Nil Rune to block that attack.
Doing so requires a block roll that uses your IN, as well as an amount of Mana (instead of Stamina) equal to 2 per AP that the Attacker used.
If you succeed, you reduce the spell or rune damage dealt by 1d6.
The amount blocked like this improves from the~\nameref{perk:runepowertraining} perk.
This action counts as a defensive Action for all other related perks.

\subsubsection{Nihilation Bolt}
For 3 AP and 4 Mana you create a bolt of negative energy and hurl it at a target in a range of~\passus{6}.
Make a ranged spell attack.
On a hit, the target takes 2d6+2 nihilation damage.