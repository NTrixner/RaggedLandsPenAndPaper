\chapter{Overview}
\section{Description}
Perks are moves, abilities and skills a character has learned, either in their past or during their adventures during play. Learning a perk requires Character Points (CP) and training time. Some perks also require a resource to study, like a book, scroll or a trainer. Perks usually look like this:

\textbf{Name}\\
The name of the perk. Perks that have multiple levels are usually followed by roman numerals detailing their level. If such a perk is noted without a level, the name usually refers to the first level of the perk.\\

\textbf{Cost}\\
The cost of the perk. A character trying to learn it has to expend this amount of character points to learn it, and they have to take a certain amount of time (in accordance to "Learning Perks" below) based on this value. A perk with a cost of "0" can be taken freely, and oftentimes either has some kind of downside or is one of a set of perks, where only one can be taken. A perk with a cost of "-" is one of the "basic starting perks", and every character is considered to have this perk, at least its first level if it has levels.

\textbf{Requirements}\\
A certain set of conditions that have to be met in order to be able to learn this perk. A character has to fulfill these requirements to be able to learn the perk.

\textbf{Tags}\\
Most perks have one of the following tags.\\
\begin{itemize}
	\item \textbf{Spell, Maneuver, Rune, Skill etc.} are types of perks. Whenever a rule mentions one of these, all perks with that type are affected by it.\\
	\item \textbf{Active/Passive} describes if a perk is usable or not. An active perk is some kind of move or spell, while a passive perk is active all of the time.\\
	\item \textbf{Repeatable} perks have multiple levels, and contain some kind of level progression which is described after the perk's main description.\\
	\item \textbf{Source required} means that a perk needs some type of source, like a trainer, a scroll, an ancient tablet in a long-forgotten language or a book. The typical gold value of such a source is given in brackets.\\
	\item \textbf{Weapon} perks are active perks that have a form of attack roll, followed by a damager roll. They are therefore treated as if they were attacks themselves. If a different perk changes an attack roll or is based upon it (like Aimed Attack, for example) this perk fulfills the requirement.\\
	\item \textbf{Memory} perks are perks that need to be studied further in order to use, even though a character may have learned it already. A character can have as many memory perks ready as their Intellect. For example, a character with an Intellect of 8 can have up to 8 memory perks remembered at a time. Usually, a character writes down their memory perks in some kind of book - maybe a religious book for prayers or a spellbook for spells. Forgetting one perk and remembering a different one takes one hour. Higher levels of the same perk, or other variations don't count as additional perks to remember.
\end{itemize}

The perk is then usually described in detail.

\section{Learning Perks}
Learning a new perk requires Character Points and time. Some perks also require a source, like a scroll, a book or a trainer. Any character that knows a perk is qualified to be a trainer for it, and can teach other characters and NPCs. 

The amount of time required to learn a new perk is based on the perk's Character Point cost and the character's intellect. It takes at least one day to train a perk, but it is possible to learn multiple perks per day. One day here is considered to consist of 16 hours of learning with frequent rests.\\

\rowcolors{2}{lightgray}{white}
\begin{tabular}{l | l | l}
	Character Intellect & CP cost covered per day & CP cost covered per hour\\
	1-3 & 100 & 6,25\\
	4-6 & 200 & 12,50\\
	7-9 & 300 & 18,75\\
	10-12 & 400 & 25\\
	13+ & 500 & 31,25
\end{tabular}