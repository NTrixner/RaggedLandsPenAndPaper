\chapter{Backgrounds}
Every character falls under one of these social statuses. It represents his standing in society, as well as some of his background. When creating a character, you have two choices: Roll a D20 to decide. When rolling, you gain all effects associated with your resulting Social Rank. If you decide instead, you just gain the social benefits or negative effects, but not the changes to your character detailed under the "Effects" column.\\

\rowcolors{2}{lightgray}{white}
\begin{tabular}{l | l | p{7.5cm}}
	Status & Roll & Effects\\ \hline
	Slave & 1 & Start with 2000 CP, +2 VIT, +2 STR\\
	Unfree & 2-3 & Start with 2500 CP, +1 VIT, +1 STR\\
	Commoner & 4-12 & None\\
	Citizen & 13-17 & Start with 3500 CP\\
	Noble & 18-19 & Start with 4000 CP, -1 VIT, -1 STR, +1 EMP, +1 INT\\
	Royal & 20 & Start with 4500 CP, -2 VIT, -2 STR, +2 EMP, +2 INT
\end{tabular}

\subsection{Slave}

Being a Slave means being unfree and uncared for. One can become a slave through war, birth or debt. A Slave has no right to be fed or to have his illnesses and wounds treated. If a Slave gets killed, it isn’t murder, but property damage. Every Slave has an owner, but sometimes they manage to escape. 

\subsection{Unfree}

Being unfree is usually the result of a high debt or a legal process (i.e. being found guilty for a heinous crime against your future owner). In some countries, it can be inherited over several generations. However, an unfree character still has his civil rights, he has a right to be fed by his owner, a right to have a safe shelter (or he just gets paid and cares for himself) and a right to be alive. Hurting an unfree worker therefore is considered a crime. However, an Unfree one still has no right to choose where he lives or works, at least until he is freed again (Usually by paying off his debt or by being unfree for a certain amount of time).

\subsection{Commoner}

Commoners are the normal people of a society. They are farmers, herders, merchants, soldiers or craftsmen. They are protected by the laws of their home country. 

\subsection{Citizen}

Citizens are people that have a home in a larger settlement or city. They therefore have additional rights (Depending on their home country), like owning slaves or being member of a guild. They don’t have to pay the tithe and have the right to own companies and partake in financial affairs. In some countries, they have the right to vote their government, administration and/or mayor

\subsection{Noble}

Nobles are either of high birth or became part of one of the many noble houses of Kusa through adoption or unrelated membership (Some noble houses take in Citizens in the same way guilds do). A noble is someone that has no right to rule, but society looks at nobles kindly. They usually make money through financial affairs or by owning company shares.

\subsection{Royal}

A royal is every noble that has a right to rule over land. This can be a small County, a medium kingdom or a giant Empire. Being a royal can be a challenging, responsible position, and directing the fortune of one or more settlements, or even whole countries, is not an easy task. Therefore, only few royals ever find their way into adventuring. However, being a royal only requires the right to rule, not the possibility. Some adventurous kings have been overthrown or suffer of amnesia and don’t even know that they should sit on a throne.