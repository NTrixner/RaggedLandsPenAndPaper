\chapter{The base mechanic}
\section{Attribute Checks}
Every check is solved by a roll of a D12. You add a specific modifier to your roll, and maybe get additional bonuses if you’re trained in that specific task. Whenever you roll a check, you compare the outcome of your roll to a Difficulty Value (DV). If your roll beats that DV, you are able to perform that task. If your value is lower than the DV, you can not perform the task. If your roll value and the DV are equal, you roll again.

\subsection{Contests}
Sometimes, two people compete in a task. In order to get the outcome, both people roll a D12 and add their relative modifiers, and optional additional dice. The one with the higher result wins the contest. If both rolls are the same, they are rolled again until they are not the same anymore.

\subsection{Retries}
In the RLP, rolls also determine ability, not (just) luck. If your character tries to disarm a trap, jump a certain distance or wants to climb a wall, your roll doesn’t determine how your character performs at that task at this moment, it determines if your character is able to perform that task under the current set of circumstances. This means, that unless the circumstances change, the outcome of the roll is fixed. Retries are not an option. If a character rolls to climb a wall and the roll determines he is not able to climb that wall, the characters have to find another way, maybe by using ropes or finding another spot to climb it.
If the circumstances change, rolls can be retried. This can mean stress, new knowledge, new abilities or a changed environment.
A character not able to jump a chasm may just be able to do it if he’s chased by a Giant Monitor Lizard.
A character who once failed to pick a lock may have learned something new about this type of lock from a book or his mentor, and that might give him the edge to pick it.
A character who wasn’t able to lift up a rock may have trained for some weeks and have gotten stronger.
A character who wasn’t able to climb the wall last night when it rained, may be able to do it in daylight, when the wall is dry.

\section{Rounding}
Since this System is written for experienced Tabletop-RPG players, there are a lot of formulas and straightforward calculations. Even though it’s noted everywhere, as a general rule, every time a division happens and the result would be a fraction when an integer is needed, you should round up.

\section{Attributes}
There are seven main attributes in the RLP:


\textbf{Strength (STR)} describes a character’s muscle strength. It’s an important ability for melee combat, and determines carrying capacity and stamina


\textbf{Vitality (VIT)} is a character’s bodily defensive ability. It determines how a character is able to withstand extreme weather conditions, sickness or poison. It’s also important for a character’s overall health.


\textbf{Dexterity (DEX)} describes a character’s hand-eye-coordination. It’s important with fine tasks like crafting or disabling traps, picking pockets and locks or tying knots. It also helps with ranged or melee combat, depending on someone’s weapon.


\textbf{Agility (AGI)} is a character’s ability to act when necessary, and also describes how a character is able to move around. It’s important for climbing, swimming, running, sneaking, gymnastics and tumbling in general.


\textbf{Intellect (INT)} describes a character’s overall rational ability, his memory and his ability to acquire knowledge. It can also be used to describe a character’s overall knowledge about a specific topic or lore. It’s also important for magic users.


\textbf{Perception (PER)} is a character’s ability to realize his surroundings. It’s important for battle, but also necessary to survive while exploring dangerous environments. It’s also important for social encounters,.


\textbf{Empathy (EMP)} describes a character’s ability to understand others. It’s important for most social encounters, and determines how easy it is for a character to haggle prices, to talk himself out of a battle or seduce a barmaid.


Each of these attributes are described as a value of 1-12, with 4-7 being average values. A character with a 2 in Empathy may not even be able to articulate a thought, while a character with an 11 in Empathy is able to persuade even kings.

\section{Pool Values}
For RLP, two main pool values are important, with a third being tracked, but really just important for magically trained characters. Every character starts with 1 level of the "Health Pool", "Stamina Pool" and "Mana Pool" Perks.


\textbf{Health} describes a character’s ability to withstand any type of damage. It can be seen as a combination of resilience, bodily health and luck. If a character reaches 0 Health, he is dead. Health can be restored by resting (See Recovering Pool values, below), by alchemy and by magic. A character that is dead can not be healed. Reviving magic doesn’t exist in the Ragged Lands.


\textbf{Stamina} describes a character’s ability to act out tasks that are bodily challenging. When a character climbs or swims fast, or if a character uses special maneuvers in combat, he loses stamina. Stamina can be restored by resting (See below). It usually doesn’t take much longer than an hour to completely restore Stamina, so outside of a stressful situation, a character usually has full Stamina. If a character reaches a Stamina of 0, he becomes unconscious. A character can force himself to continue even when reaching a Stamina of 0 by making a Vitality check with an initial DV of 10. For every round this goes on, however, the character loses 1D6 health, and has to repeat the check, with the DV increasing by 1 every round.


\textbf{Mana} is usually tracked for every character, but only important for magic users. It describes a character’s ability to cast spells. Casting a spell costs Mana, and a character replenishes mana by eating. A character that reaches 0 Mana becomes paralyzed.


\subsection{Relevant Attributes}

The 3 Pool Values are based on different attributes. Health is based on Vitality, Stamina is based on Strength and Mana is based on Intellect. Whenever these attributes change, the Pool values change accordingly. For example. a character may have 9 Intelligence. Therefore, this character starts playing with 9 Mana plus the value of their initial D6 roll (let's say 4, making it a total of 13), assuming the player didn’t invest in the second “Increase Mana” Perk. If the player now increases the IN value by 1, the character now has 10 IN, which increases the character’s Mana automatically to 14.

If the character now gains the “Increase Mana II” perk and rolls a 4, they gain 14 Mana and are at a total of 28.

Now, the character may be hit by an Intelligence-draining attack, bringing their IN down to 8 (A reduction of 2 points). Since their IN was assigned 2 times to their Mana (Once for each level of "Increase Mana"), their Mana is reduced by 4, to a total of 24.


This mechanic is the same for all 3 Pool Values and their corresponding attributes.


\subsection{Recovering Pool Values}

For each of the 3 different Pool values, recovery works the same way, but is triggered by different conditions. Whenever one of these conditions occur, the characters regain points in that pool value as mentioned below


\subsection{Health}

Aside from using special tools or magic for healing, each character has the ability to recover health in a natural way. A character's natural healing ability allows them to heal 1 points of health, per 1 hour.
A priest or healer may improve someone's natural healing ability.


\subsection{Stamina}

Stamina can be recovered by taking a break, regaining breath, not moving too much around, sitting down for a minute or two or drinking a bit of water. For every 3 seconds a character rests like that (or every AP in combat), they gain back 1 Stamina.



\subsection{Mana}

Mana can be recovered by eating. Every time a character eats a full meal's worth of food, they regain 1d6 points of Mana. In the equipment section, there is an item called "Ration per day" - this is the equivalent of 3 meals.

Better foods may add bonuses to this die roll.