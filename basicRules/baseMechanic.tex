\chapter{The Base mechanic}\label{ch:baseMechanic}


\section{The Meta Mechanic}\label{sec:metaMechanic}
At the beginning of a scene, the GM describes the scene, as well as any points of interest.
The Players then decide what their player characters want to do.
These actions are then adjudicated.
After actions are adjudicated, the scene has changed and the GM describes the changed scene, or the next scene.

\subsection{Action Adjudication}\label{subsec:actionAdjudication}
Whenever an action happens, it is adjudicated.
The GM decides two things: A) Can the action succeed.
If the action is impossible to succeed, the GM should usually tell the Player.
B) Can the action fail.
If an action can not fail, there is no need to make a roll.\\
If an action is both possible to fail as well as possible to succeed, a character must make an appropriate check.
Most of these checks are Attribute Checks, as described in the next section.\\


\section{Attribute Checks}\label{sec:attributeChecks}
Every check is solved by a roll of a D20 You add a specific modifier to your roll, and maybe get additional bonuses if you're trained in that specific task.
Whenever you roll a check, you compare the outcome of your roll to a Difficulty Value (DV).
If your roll beats that DV, you are able to perform that task.
If your value is lower than the DV, you can not perform the task.
If your roll value and the DV are equal, you roll again.

\subsection{Contests}\label{subsec:contests}
Sometimes, two people compete in a task.
In order to get the outcome, both people roll a D20 and add their relative modifiers, and optional additional dice.
The one with the higher result wins the contest.
If both rolls are the same, they are rolled again until they are not the same anymore.

\subsection{Helping Others and working together}\label{subsec:helping}
Sometimes, a character may be able to help another character with a difficult task.
In such a case, the player describes how exactly they are helping, and the GM adjudicates if it's one of three general scenarios.\\
\textbf{Guided acting} occurs if one character tells another character what to do in order to achieve a task.
In this case, the acting character may add +2 to the roll for every rank of Skill Proficiency Perk that the guiding character has for said task.\\
For example, a character who is trained in picking locks explaining a different character how to pick a certain lock while the second character is actively doing it might give the lock picker +4 bonus if they have 2 ranks in the~\nameref{sec:lockPicking} perk.\\
\textbf{Direct Help} If a character helps another character directly in a certain task, the helping character may have to roll for the specific way in which they are helping.
If they succeed, the Difficulty Value of the original task is reduced.\\
For example, if a character is failing their check to climb a cliff, another character who is already at the top might reach down and help them up.
In this case, a strength check might be required from the helper, and would ease the climbing check for the failing character.\\
\textbf{Working together} If two or more characters are working together, every character makes a roll.
Depending on the nature of the task, either the group succeeds if one of them succeeds, or the group fails if one of them fails.\\
Depending on the task, working together might change the DV of the given task.\\
For example, a group of people sneaking past a monster might fail if one of them fails their check.
It might also make it more difficult to sneak past a monster as a group.
However, a group of people pushing against a rock to move it might succeed if only one of the group succeeds, and the DV would probably be lowered.

\subsection{Retries}\label{subsec:retries}
In the RLP, rolls also determine ability, not (just) luck.
If your character tries to disarm a trap, jump a certain distance or wants to climb a wall, your roll doesn't determine how your character performs at that task at this moment, it determines if your character is able to perform that task under the current set of circumstances.
This means, that unless the circumstances change, the outcome of the roll is fixed.
Retries are not an option.
If a character rolls to climb a wall and the roll determines they are not able to climb that wall, the characters have to find another way, maybe by using ropes or finding another spot to climb it.
If the circumstances change, rolls can be retried.
This can mean stress, new knowledge, new abilities or a changed environment.
A character not able to jump a chasm may just be able to do it if he's chased by a Giant Monitor Lizard.
A character who once failed to pick a lock may have learned something new about this type of lock from a book or his mentor, and that might give him the edge to pick it.
A character who wasn't able to lift up a rock may have trained for some weeks and have gotten stronger.
A character who wasn't able to climb the wall last night when it rained, may be able to do it in daylight, when the wall is dry.


\section{Rounding}\label{sec:rounding}
Since this System is written for experienced Tabletop-RPG players, there are a lot of formulas and straightforward calculations.
Even though it's noted everywhere, as a general rule, every time a division happens and the result would be a fraction when an integer is needed, you should round up.

\section{Modifiers Types}\label{sec:modifierTypes}
When modifiers are applied to bonuses and maluses, they usually have a type associated with them.
For example, a strength potion might give a character a temporary "alchemical" bonus to strength.
In general, a creature only profits the strongest bonus of a given type, and only suffers the strongest malus of a given type, but different types of bonuses and maluses are summed up.
For example, a creature might gain a +2 Alchemical Strength bonus and a +3 Magical Strength bonus, gaining a total of a +5 to Strength.
However, if the same creature gains a second Alchemical Strength bonus with +4, the total bonus comes to +7, not +9, since the creature stops profiting from the +2 Alchemical Strength bonus (While still having it - if the +4 bonus ends earlier, the total goes back to +5).
On the other end of the spectrum, a creature suffering a -5 Magical Strength Malus and a -2 Magical Strength Malus would only suffer a total of -5.
Additionally, some bonuses and maluses apply dice instead of flat values.\\
For example, being trained in a skill will provide a +1d4, +2d4 or +3d4 skill bonus.
In such cases, the largest amounts of the same sized dice (both positive and negative) apply to the checks.
Additionally, positive and negative dice of equal size cancel each other out (for example, a creature with +1d4 and -1d4 has a bonus of 0).
For example, a creature might have a +2d4 skill bonus, a +1d4 skill bonus, a +1d4 racial bonus, a +1d6 skill bonus, a +5 skill bonus, a -1d4 profane malus, a -2d4 profane malus, and a -1d6 profane malus on the same roll.
As a result, that creature would have +3d4 (since the racial and skill 1d4 bonuses stack, but not both skill bonuses), +1d6, +5, -2d4 (again, both profane maluses don't stack), and -1d6 (different dice, therefore they stack).
The total would be +1d4 + 5, since the +3d4 and -2d4 reduce to 1d4, the two 1d6 cancel each other out, and the +5 is left over.\\
Lastly, some modifiers are multiplicative.
Multiple multiplicative modifiers of different types are treated as if they were themselves summations.
For example, two different x2 modifiers are treated as if each one was a multiplier bonus of +1, therefore resulting in a total of x3.\\
In general, multipliers are applied first, fractions second, and then dice bonuses/maluses and flat bonuses/maluses.\\

\section{Example DVs}\label{sec:exampleDVs}
Here is a list of example difficulty values.
These are designed for a standard group of beginner adventurers.
If the DV is set by something or someone that is trained in what they are doing, you might add the respective level to that DV.
For example, a trap set by someone of average skill might have a Difficulty Value of 13.
However, the same trap made by someone who is skilled at trap-making might be closer to 15, if that trap-maker has a crafting level of 2.
If such a trap would be made by someone who is well-trained too, the Difficulty Value might even increase to something in the range of 16--19.
Circumstantial bonuses might increase or decrease a DV by 1 to 10.

\begin{itemize}
    \item Very easy: 3\\
    These checks are usually not necessary, unless there's significant circumstantial hindrances.
    \item Easy: 8\\
    Most characters should be able to succeed on these checks.
    Trained characters will almost always succeed on these checks.
    \item Medium: 13\\
    Most characters have a good chance of succeeding on these checks.
    Trained characters will usually succeed on these checks.
    \item Hard: 18\\
    These tasks are becoming more and more difficult.
    Only well-trained characters and seasoned adventurers have a high likelihood to succeed.
    Others will have to be very lucky.
    \item Very Hard: 23\\
    These tasks are becoming extremely difficult.
    It has become very unlikely that someone untrained would be able to succeed on this task.
    Those trained in the task are also more and more likely to fail.
    \item Near impossible: 28\\
    This task is impossible for anyone who is below average in the given attribute.
    Even for the well-trained, it is more likely than not that they will fail.
\end{itemize}

\section{Passage of Time}\label{sec:passageOfTime}
Actions usually take at least some time.
A GM will adjudicate an action at the moment that it starts, and then time will pass while they are acting.\\
The amount of time passing should be clear to the players, and the GM should always be aware what character is performing which action.\\
The amount of time passing during an action is determined by the action itself, and varies during different frames of gameplay.
For example, overland travel might be tracked in hours or even days, while searching a room might take minutes.
All the while, combat will only take a few minutes, maybe even just seconds.

\subsection{Action Points}\label{subsec:actionPoints}
Action Points are a special unit used within the rest of the document.
Action Points (\textbf{AP}) are used to measure time in action-heavy scenes.
An Action Point equals to around 2 seconds.


\section{Attributes}\label{sec:stats}
There are seven main attributes in the RLP:\\

\textbf{Strength (ST)}\label{stat:strength} describes a character's muscle strength.
It's an important ability for melee combat, and determines carrying capacity.\\


\textbf{Vitality (VI)}\label{stat:vitality} is a character's bodily defensive ability.
It determines how a character is able to withstand extreme weather conditions, sickness or poison.\\


\textbf{Dexterity (DE)}\label{stat:dexterity} describes a character's hand-eye-coordination.
It's important with fine tasks like crafting or disabling traps, picking pockets and locks or tying knots.
It also helps with ranged or melee combat, depending on the attacker's weapon.\\


\textbf{Agility (AG)}\label{stat:agility} is a character's ability to act when necessary, and also describes how a character is able to move around.
It's important for climbing, swimming, running, sneaking, gymnastics and tumbling in general.\\


\textbf{Intellect (IN)}\label{stat:intellect} describes a character's overall rational ability, his memory and his ability to acquire knowledge.
It can also be used to describe a character's overall knowledge about a specific topic or lore.
It's also important for magic users.\\


\textbf{Perception (PE)}\label{stat:perception} is a character's ability to realize his surroundings.
It's important for battle, but also necessary to survive while exploring dangerous environments.
It's also important for social encounters.\\


\textbf{Empathy (EM)}\label{stat:empathy} describes a character's ability to understand others.
It's important for most social encounters, and determines how easy it is for a character to haggle prices, to talk himself out of a battle or seduce a barmaid.\\
Each of these attributes are described as a value of 1--12, with 4--7 being average values.
A character with a 2 in Empathy may not even be able to articulate a thought, while a character with an 11 in Empathy is able to persuade even kings.

\section{Defensive Attributes}\label{sec:defensiveStats}
These values are derived from the main Attributes and describe defensive values that are used when a character would be target of some effect.\\
Each and every one of these Attributes consists of three values: The bonus, the lower attribute and the higher attribute.\\
The Lower Attribute is equal to 8 + the Bonus.\\
The Higher Attribute is equal to 21 + the Bonus.\\
A Check with a Defensive Attribute is equal to 1d20 + the Bonus.\\
When resolving an effect that requires a roll against a defensive attribute, the effect fails if the roll lower than the lower attribute, succeeds if it's higher than the higher attribute, and allows the creature to act defensively (by making a roll with the bonus) if it is between.
If the effect doesn't require a roll by whoever is causing it, but has a fixed Difficulty Value, the target can always make a check against it (but it doesn't really need to if the Difficulty Value is higher than the Higher Value, since that would require a die roll higher than a 20).\\

For example, if a creature attacks another creature, the attacker makes an attack roll against the Dodge of the defender.
If the attack roll is below the defender's Lower Dodge, the attack automatically misses.
If it's above the defender's Higher Dodge, it automatically hits.
If it's in between those two values (including the values themselves), the defender can try to use a Dodge action (see combat section) to make a contested roll against the attack roll, using their Dodge Bonus.\\

By default, each Defensive Attribute is equal to one Main Attribute.
However, characters can learn Perks that increase their bonuses or gain temporary bonuses from magic items, spells or potions, or other sources.\\


\textbf{Dodge}\label{stat:dodge} is a character's ability to dodge an incoming attack.\\
The bonus for this attribute is based on Agility.\\

\textbf{Notice}\label{stat:notice} is a character's ability to notice something unseen, like a noise, slight draft, or sneaking enemy.
It does not automatically unveil anything, but a hidden object not reaching a character's notice value will at least give them the idea that something is off.\\
The bonus for this attribute is based on Perception.\\

\textbf{Willpower}\label{stat:willpower} is a character's ability to withstand emotional manipulation, as well as enchantment.
If a character is target to emotional manipulation or a spell or magical effect they don't want to be affected by, and the roll associated with the effect does not reach their Willpower value, the resist the effect outright and don't have to roll competing checks.\\
The bonus for this attribute is based on Empathy.\\

\section{Pool Values}\label{sec:pools}
For RLP, two main pool values are important, with a third being tracked, but really being just important for magically trained characters.
The base number of these values is determined by a character's race and attributes.\\

\subsection{Relevant Attributes}\label{subsec:relevantAttributesPools}
The three Pool Values are each dependent on an Attribute.
Health depends on Vitality, Stamina depends on Strength, and Mana depends on Intellect.
This means that each of the Pool Values is equal to a number stemming from a race, plus the relevant attribute.
For example, a Daevana starts with 25 Health.
With 9 Vitality, a Daevana would start the game with 34 Health.
If that character then increases their Vitality by 1, their maximum Health would also increase.
Additionally, with each rank in ~\nameref{perk:increasehealth}, Vitality is added to the character's maximum health again.
This also means that a character with 3 ranks in Increase Health would gain 4 additional maximum Health if they increase their Vitality by 1.

\subsection{Health}\label{pool:health} describes a character's ability to withstand any type of damage.
Health can be seen as a combination of resilience, bodily health and luck.
If a character reaches 0 Health, they are \textbf{dead}.
A character with less than one fourth of their Health becomes "\nameref{condition:wounded}".
A character with less than one tenth of their Health becomes "\nameref{condition:heavilyWounded}".
Health can be restored by resting (See Recovering Pool values, below), by alchemy and by magic.
A character that is dead can not be healed.
Reviving magic doesn't exist in the Ragged Lands.\\

\subsection{Stamina}\label{pool:stamina} describes a character's ability to act out tasks that are bodily challenging.
When a character climbs or swims fast, or if a character uses special maneuvers in combat, they loses stamina.
Stamina can be restored by resting (See below).
It usually doesn't take much longer than an hour to completely restore Stamina, so outside of a stressful situation, a character usually has full Stamina.
A character with less than one fourth of their Stamina becomes "\nameref{condition:exhausted}".
A character with less than one tenth of their Stamina becomes "\nameref{condition:heavilyExhausted}".k
If a character reaches a Stamina of 0, they become \textbf{unconscious}.\\

\subsection{Mana}\label{pool:mana} is usually tracked for every character, but only important for magic users.
It describes a character's ability to cast spells.
Casting a spell costs Mana, and a character replenishes mana by eating.
A character with less than one fourth of their Mana becomes "\nameref{condition:drained}".
A character with less than one tenth of their Mana becomes "\nameref{condition:heavilyDrained}".
A character that reaches 0 Mana becomes \textbf{paralyzed}.\\

\subsection{Recovering Pool Values}\label{subsec:recoveringPoolValues}
For each of the 3 different Pool values, recovery works the same way, but is triggered by different conditions.
Whenever one of these conditions occur, the characters regain points in that pool value as mentioned below

\subsection{Recovering Health}\label{subsec:recoverHealth}
Aside from using special tools or magic for healing, each character has the ability to recover health in a natural way.
A character's natural healing ability allows them to heal 1 points of health, per 1 hour.
While resting, this healing effect is doubled.
A priest or healer may further improve someone's natural healing ability.

\subsection{Recovering Stamina}\label{subsec:recoverStamina}
Stamina can be recovered by taking a break, regaining breath, not moving too much around, sitting down for a minute or two or drinking a bit of water.
For every 2 seconds a character rests like that (or every AP in combat they spend just resting), they gain back 1 Stamina.

\subsection{Recovering Mana}\label{subsec:recoverMana}
Mana is recovered over time, but the rate at which it recovers can be influenced by the food that the character eats.
A meal usually lasts for 8 hours.
A usual ration provides a recovery of 1d6 points per hour.

\subsection{Temporary Pool Values}\label{subsec:temporaryPoolValues}
When a character gains temporary bonuses to their pool values, they function as temporary increase of their maximum values.\\
For example, a character that has 20 out of 30 Health, and gains 10 temporary health, has 30 out of 40 Health.
After the effect ends, they would lose those temporary Health, and their current Health would also be reduced by the same amount.
If that character had taken more than 10 damage in the meantime, this would mean that they would succumb to their wounds once the temporary Health run out.\\