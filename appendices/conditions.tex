\chapter{Conditions}

\section{Calmed}
A calmed character can dodge attacks even if they exceed his Reaction Value, and dodging doesn't cost stamina for a calmed character. Taking damage usually ends this condition.

\section{Enraged}
An \textbf{enraged} character adds their Strength to any Strength-based attack damage. They lose their ability to do anything else than moving and attacking with melee, Strength-based weapon attacks. This means they also cannot dodge, block, or use attack-augmenting perks like Aimed Attack or Brutal Attack. They drop any wielded weapons that don't allow for such an attack and use their fists instead. They then move to the closest creature no matter if friend or foe and start trying to kill it (choosing at random if multiple creatures are the same distance away). While enraged, a character loses 1 point of stamina per AP, until they fall unconscious. If an enraged character reaches 0 Health, they don't die immediately. Instead, any damage beyond 0 Health reduces their stamina. If they aren't healed until their rage ends, they die once it does.

\section{Entranced}
An entranced character is only able to perceive the source of their entrancement. Any other Perception checks automatically fail. Taking damage usually ends this condition.

\section{Frightened}
A frightened character can not move freely towards the cause of their fear. If the effect has no specific cause, they can only cower. While frightened, a character suffers from a -5 penalty on attack rolls. If the source of a character's fear is removed, this effect usually ends.

\section{Hastened}
A hastened character can act twice as fast as normal. This means that any action's AP is reduced by half, to a minimum of 1. Hastened and Hindered cancel each other out completely. A hastened character who is affected by the Hindered effect loses both conditions.

\section{Hindered}
A hindered character can act half as fast as normal. This means that any action's AP is doubled. Hastened and Hindered cancel each other out completely. A hindered character who is affected by the Hastened effect loses both conditions.

\section{Invisible}
An invisible character can not be seen. When an invisible character tries to move stealthily, they can roll twice, taking the higher result. Any attack rolls against invisible characters are reduced by 10.

\section{Poisoned}
A poisoned character suffers from the effects of a specific poison. The poison itself usually states what effect this is, and how long the poison lingers.

\section{Prone}
A prone character is crawling or lying on the ground. Melee attack rolls against prone characters are increased by 5, ranged attack rolls against prone characters are reduced by 5. While being prone, moving costs twice the AP for a character.
Standing up is a form of movement, and takes 2 AP if the character is unarmored or in light armor, 4 AP if the character is in medium armor and 6 AP if they are in heavy armor.

\section{Restrained}
A restrained character can not move, and sometimes is also unable to take other actions. Breaking oneself free is usually a strength check against the source of restriction.

\section{Sleeping}
A sleeping character is usually considered prone. A sleeping character suffers from a -10 penalty on perception checks and is otherwise considered unconscious. Loud noises, being moved or taking damage end this condition immediately. However, a character can make a vitality check to force themselves to sleep in harsh conditions. 

\section{Paralyzed}
A paralyzed character is unable to move or act, and sometimes also fall prone. Paralysis usually ends after some time, or when the causing effect ends.

\section{Unconscious}
An unconscious character can not react, is unable to move and usually prone. An unconscious character can not move or act, doesn't realize what happens around them and automatically fails all checks that require action, perception or movement.
