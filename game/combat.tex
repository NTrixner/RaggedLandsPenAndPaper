\chapter{Combat}
\textit{“The smell of sweat and blood enters your nose, as the noise of steel on steel draws closer to you. You can hear shouts, human voices that try to organize each other’s actions. The dust blocking your sight slowly settles. The soldiers have heavily wounded the giant, which has entered a frenzied state, blood and acid dripping from its mouth. The giant’s club, which resembles the trunk of a pine tree, smashes down on one of the warriors, burying him with an uncanny, crunching sound. He didn’t get a chance to scream.”}


Most, if not all, Role Playing Games include some form of combat. Combat is an easy way to build excitement in an adventure. When words failed, or weren’t even an option to begin with, when your character enters in a gladiator’s competition, even when two of your characters are just friendly sparring with each other, combat is happening.


\section{Attribute use}

\textbf{Strength} is used for most melee attacks.\\~\\
\textbf{Vitality} is used to determine your overall health, if you’re able to resist poisoned weapons, and it is important for casters that need to concentrate on a spell.\\~\\
\textbf{Dexterity} is used for most ranged attacks and some melee attacks.\\~\\
\textbf{Agility} is used to define how many actions a character can take, and it’s important for dodging. Also, unarmed combat can be based on Agility\\~\\
\textbf{Intellect} is used for spells that can be cast in battles. It’s also used to determine how much a character knows about the abilities an enemy has.\\~\\
\textbf{Perception} is important for most reactions, and a character with a high perception value is harder to surprise. Also, a character with high perception can perform more actions.\\~\\
\textbf{Empathy} may not sound like it’s important in combat. However, some perks require Empathy checks to try and control enemy behavior (The “Provoke” perk, for example)

\section{Combat basics}
Whenever a combat situation erupts, the GM starts the battle by deciding if one or multiple participants are surprised. These combat participants are not allowed to act until the others have acted. Then, the GM lets the characters act in a specific order that he decides is the most fitting.

Every action in combat takes a specific amount of Action Points. Action Points are a measure of time, with one Minute consisting of about 100 Action Points. After an action was resolved, the acting character has to wait for the amount of Action points his action took before being able to act again. If no participant is acting, the GM counts down Action Points until a participant is again able to act.

As an alternative approach, an experienced GM may decide to not use AP at all. This can make the game more flowing and immersive, but it may prove to be difficult to balance.



\section{Actions in combat}

\begin{itemize}

\item Attack [AP depending on weapon]

In order to attack an opponent, the character has to be in reach of said opponent. This reach is defined by the character’s weapon. He then has to make an attack roll:

Attack roll = 1D12 + STR/DEX/AG + [Combat Level + XD4] + misc

The Combat level and D4s can only be added if the character is trained to a certain amount in the Weapon Training perk of the weapon they are using.

If the attacker has an attack roll that is higher than the defender's Reaction Value, the defender can not react to the attack. This is also the case if the defender is unaware of the attacker. Otherwise, the defender can react in one of three ways: Dodge, Block or take the hit. All of these actions - except taking the hit - require a successful check and a specific amount of Stamina.
The defender's Reaction Value is equal to:

RV = 7 + PE + [Combat Level + X*2] + misc

The Combat level and 2s can only be added if the character is trained to a certain amount in the Armor Training perk of the armor they are wearing.

The amount of stamina used for a defensive action is based on the amount of damage dice that the attacker would get, including additional dice from different sources. For example, slashing with an arming sword deals 1d8 points of damage. Dodging costs 2 Stamina per damage die, therefore it costs 2 Stamina to dodge a slash from an arming sword. If the attacker uses the Brutal Attack Perk, however, they can add damage dice to the roll. When adding a second damage die before the roll, blocking the attack becomes more difficult, costing 4 Stamina instead of 2.

The DV for the defensive check is the attack roll. The defensive action may prevent any damage from happening. Otherwise, the attacker rolls damage according to his weapon. The defensive action and the defender's armor may reduce that damage, and the rest is dealt to the defender’s health.



\textbf{Fumbles and Critical hits}

When you roll a Natural 12 on a Weapon attack roll, you automatically hit as if the defender would take the “Take Hit” action and deal damage as if the defender wasn’t wearing armor. Also, there is a chance to wound the opponent. The attacker rolls a 1d6 to decide which body part they hit and roll damage normally. If the damage dealt is larger than the wound threshold that the defender's armor at that body part provides, a wound of the weapon's damage type is caused. If the weapon deals multiple types of damage, each type of damage is treated differently for wounding. Therefore each type of damage has to beat the threshold, but each multiple types of damage may cause multiple wounds with one attack. See below for different wound types and their effects. 

This is called a “critical hit”, and some enemies are immune to it or have a chance to resist a critical hit.

When you roll a Natural 1 on a Weapon attack, you automatically miss your attack.



\item Dodge [Defensive; 2 Stamina per attacking Damage Die]

Dodging requires an Agility check against the attack roll. If the defender is trained in the armor they are wearing, they can add their Combat Level to this check. If the defender succeeds this check, they move one passus to a free spot and take no damage. If there is no free spot, or the defender’s roll failed, they take normal damage instead. Dodging can be used against melee and ranged attacks.



\item Block [Defensive; 1 Stamina per attacking Damage Die]

Blocking requires a Weapon defense roll against the attacker's attack roll. If the defender succeeds this check, they roll a weapon defense damage roll and add the result to their armor rating before reducing the attacker’s damage by the total. (When wielding two weapons or a weapon and a shield, the defender may choose one of the two). Blocking can be used against melee and ranged attacks, but ranged attacks can only be blocked by shields.



\item Take Hit [Defensive; 0 Stamina]

Taking the hit is the default defensive action, therefore it doesn’t cost anything. The attacker automatically hits, and when he didn’t roll a critical hit (A Natural 12), the weapon damage is reduced by the defender’s Armor Rating. Taking the hit can be used against melee and ranged attacks.



\item Counter [Defensive; 3 Stamina per own Damage Die]

When a defensive action (except taking the hit) reduces a melee attack's damage to 0, the defender can use the opportunity given by the botched attack to counter it with their own attack. Doing so is quite strenuous, requiring 3 Stamina Points per damage die of the counter-attack.



\item Move [1 AP per passus]

In order to move in a strategic manner, the character has to pay 1 AP per passus (approximately 1.5m, or 5 feet) moved. 



\item Run  [1 AP per 2 passus]

In order to run to a specific position, the character loses 1 stamina per 2 passus. If two characters try to run to the same spot at the same time, an Agility-check decides who arrives first. If one of them is further away from their goal, this contestant gains a malus of 1 for every passus of difference.
\end{itemize}

\section{Wounds}
Every character has a wound threshold, which is equals to their Vitality + a bonus provided by armor and other sources. Whenever an attack deals more damage than the target's wound threshold, a wound is caused. It is applied to a random body part (roll a d6 and consult the table below). The amount by which it exceeds the wound threshold is known as the wound's value. For example, a character with a wound threshold of 3 receiving a Torso wound by being dealt a critical hit for 5 cutting damage would take 2 damage per minute. If the wound was aimed at the head, it would deal 4 damage per minute instead.\\
A wound is typed, and if an attack deals multiple different damage types, they are treated as different attacks for this purpose.

The effects of wounds are cumulative, and after a body part receives 3 wounds (of any type), the body part becomes crippled, and the Crippled Condition of that body part is applied.\\

\rowcolors{2}{lightgray}{white}
\begin{tabular}{l p{1cm} p{5cm} p{3cm}}
d6 & Body Part & Wound effect & Crippled effect\\ \hline
1 & Torso & Damage per minute & Max/Current Health halved\\
2 & Left Leg & -2 dodge rolls, +1 AP move/run & -4 Agility (min 1), Character loses Left Leg\\
3 & Right Leg & -2 dodge rolls, +1 AP move/run & -4 Agility (min 1), Character loses Right Leg\\
4 & Left Arm & -2 on Attack/Defense rolls with this arm & -2 Strength, -2 Dexterity. Character loses Left Arm\\
5 & Right Arm & -2 on Attack/Defense rolls with this arm & -2 Strength, -2 Dexterity. Character loses Right Arm\\
6 & Head & 2x Damage per minute & Death
\end{tabular}
