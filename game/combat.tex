
\chapter{Combat}
Whenever a combat situation erupts, the GM starts the battle by deciding if one or multiple participants are surprised.
These combat participants are not allowed to act until the others have acted.
Then, the GM lets the characters act in a specific order that he decides is the most fitting.

Every action in combat takes a specific amount of Action Points.
Action Points are a measure of time, with one Minute consisting of about 30 Action Points.
After an action was resolved, the acting character has to wait for the amount of Action points his action took before being able to act again.
If no participant is acting, the GM counts down Action Points until a participant is again able to act.

As an alternative approach, an experienced GM may decide to not use AP at all.
This can make the game more flowing and immersive, but it may prove to be difficult to balance.


\section{Actions in combat}

\subsection{Combining Actions}
Broadly speaking, actions are divided into two categories -- movement actions and static actions.
Whenever a character acts, they can spend up to as many AP in movement actions as they have Agility, in addition to making one static action.
After doing so, they cannot act until as many AP have passed as the sum of actions they took cost.

For example, a character with 7 Agility may dash \passus{14}, paying 7 Stamina in the process, and then do an attack that costs 4 AP.
Afterwards, they can not act again for 8 AP.

\subsection{Static Actions}

\subsubsection{Offensive}
\begin{itemize}
\item Attack [AP depending on weapon]
In order to attack an opponent, the character has to be in reach of said opponent.
This reach is defined by the character's weapon.
He then has to make an attack roll:\\
\\
Attack roll = 1D20 + STR/DEX/AG (depending on the weapon) + [Combat Level + XD4] + misc
\\
The D4s and level can only be added if the character is trained to a certain amount in the attack they are making.
If the attacker has an attack roll that is lower than the defender's Dodge, the attack misses.
If the defender is unaware of the attack, they can not react.
Otherwise, the defender can react in one of three ways: Evade, Block or take the hit.
All of these actions - except taking the hit - require a successful check and a specific amount of Stamina.
The amount of stamina used for a defensive action is based on the amount of  AP that the attack costs.
The DV for the defensive check is the attack roll.
The defensive action may prevent any damage from happening.
Otherwise, the attacker rolls damage according to his weapon.
The weapon entry states what attribute is added to the damage roll, and how much of it is added.
When calculating this value, remember to round up.
The defensive action and the defender's armor may reduce the resulting damage, and the rest is dealt to the defender's health.
If the attack deals multiple types of damage, each type of damage is treated separately.


\textbf{Fumbles and Critical hits}
When you roll a Natural 20 on a Weapon attack roll, you automatically hit as if the defender would take the "Take Hit" action and deal additional damage.
This means that any weapon damage die is rolled twice, taking the total result when determining the damage of the attack.
This is called a "critical hit", and some enemies are immune to it or have a chance to resist a critical hit.
When you roll a Natural 1 on a Weapon attack, you automatically miss your attack.


\item Trip [AP depending on weapon; 10 Stamina]
Make a melee attack roll against a standing opponent, using an additional 5 Stamina.
Your attack roll is contested by the enemy's Evade or Block Action.
This defensive action does not cost any Stamina.
For every foot (or other type of grounding appendage) the enemy has, they gain +1 on their roll.
When your enemy fails, they drop prone.\\
Unlike a normal attack, a trip does not deal damage.\\


\item Disarm [AP depending on weapon; 10 Stamina]
You attack one opponent that is armed with a manufactured weapon.
Make a melee weapon attack roll, taking an additional 5 Stamina, opposed by the enemy's weapon attack roll.
If you succeed, the enemy drops their weapon to the floor.
You don't deal damage with this attack.\\


\item Counter [2 Stamina per counter attack AP]

When a defensive action (except taking the hit) reduces a melee attack's damage to 0, or if an attacker rolls a natural 1 on a melee attack, the defender can use the opportunity given by the botched attack to counter it with their own attack.
Doing so is quite strenuous, requiring 3 Stamina Points per AP cost of the counter-attack.


\item Fight defensively [1 AP]

A character can decide to fight defensively for the duration of one Action Point.

\item If they do so, they gain +4 on defensive rolls and +2 on Dodge for that duration.
A character may declare to fight defensively for a specific amount of time, or until a certain condition is met.
In the latter case, they act again on the AP count after that condition triggers.

\subsubsection{Defensive}


\item Evade [Defensive; 3 Stamina per attack AP]

Evading requires an Agility check against the attack roll.
If the defender succeeds this check, they move \passus{1} to a free spot and take no damage.
If there is no free spot, or the defender"'s roll failed, they take normal damage instead.
Evading can be used against melee and ranged attacks.

\item Block [Defensive; 2 Stamina per attack AP]

Blocking requires a Weapon defense roll against the attacker's attack roll.
If the defender succeeds this check, they roll a weapon defense damage roll and add the result to their armor rating before reducing the attacker's damage by the total.
(When wielding two weapons or a weapon and a shield, the defender may choose one of the two).
Blocking can only be used against melee attacks unless otherwise stated by the blocking weapon.

\item Take Hit [Defensive; 0 Stamina]

Taking the hit is the default defensive action, therefore it doesn't cost anything.
The attacker automatically hits, and the weapon damage is reduced by the defender"'s Armor Rating.
Taking the hit can be used against melee and ranged attacks.


\subsection{Movement Actions}

\item Move [1 AP per \passus{1}]

In order to move in a strategic manner, the character has to pay 1 AP per \passus{2} moved.


\item Run [1 AP and 2 Stamina per \passus{2}]

In order to run to a specific position, the character loses 2 stamina per \passus{4}.
If two characters try to run to the same spot at the same time, an Agility-check decides who arrives first.
If one of them is further away from their goal, this contestant gains a malus of 1 for every passus of difference.


\item Draw/Sheathe Weapon [AP cost depends]

When not armed, a character first has to draw their weapon.
Also, if they want to change weapons, they usually have to either drop the weapon they're wielding (which doesn't cost any AP) or sheathe it, before drawing a different weapon.
The AP to do so is specified by the weapon.


\end{itemize}