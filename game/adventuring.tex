\chapter{Adventuring}\label{ch:Adventuring}
\section{Time, Speed and Travel}\label{sec:timeSpeedTravel}
\subsection{Time Units}\label{subsec:timeUnits}
The relations between seconds, minutes, hours and days are equivalent to earth units, even though they are somewhat longer units overall, since Kusa, the planet of the Ragged lands, is larger than earth.\\

A week on Kusa has 10 days, which are usually numbered (first weekday, second weekday, etc.), and the last 3 days are considered to be weekend days - even though that doesn't have much effect on common folk, who have to work in the fields and care for their animals anyway.\\

Each Kusaen Month consists of three weeks, i.e.\ 30 days.\\

A year begins mid-winter and consists of 13 months and one additional week.

During this week, also called "black week" or "dark week", the unique constellation of the planet's two moons cause a week-long eclipse, which shrouds the planet in complete darkness.
In this time, areas outside of towns are filled with monsters that disappear once the sun rises again.
This is also the point at which the new year starts.\\

\subsection{Short-Distance Travel}\label{subsec:shortDistanceTravel}
During Combat or in situations where split-second decisions are important, a character is assumed to be able to tactically move \passus{2} per AP, or run \passus{4} per AP and Stamina spent.
If a character is trying to move stealthy, they are considered to be half as fast during combat or during a time-based encounter.\\

\subsection{Mid-Distance Travel}\label{subsec:midDistanceTravel}
While travelling in areas with a dense layout, like a town, city or a cavern, a character is considered to be able to move \passus{50} per minute.

\subsection{Long-Distance Travel}\label{subsec:longDistanceTravel}
While travelling overland, a character is considered to be able to move \leg{4} per hour, which means that a well-travelled character should be able to move \leg{16} per 4-hour interval;
Adventurers are usually considered to be well-travelled.\\

One travelling day is sectioned into six separate 4-hour intervals, three of which are usually spent travelling.

Travelling more than 16 hours reduces a character's maximum Stamina by 1 for each additional hour they spend travelling until they take a rest, at which point one maximum Stamina is restored for each hour rested.\\

For each 4-hour interval, each character in the group can choose a side activity.

These consist of:

\begin{itemize}
	\item Follow Tracks

	\item Ensuring that the group moves stealthy (reduces the group's speed by half)

	\item Lookout for ambushes

	\item Scavenge the area for items

\end{itemize}
This list is not exhaustive.\\

Depending on the area of travel and the group's makeup, there could be more options.
For example, if the group is moving by horse-drawn wagon, one member of the group has to lead the wagon, while others could use the wagon to rest, learn a perk or even craft items.

\subsection{Stealth}\label{subsec:stealth}
If a creature tries to move stealthy in a dungeon, they are considered to be able to move at one-fourth speed, meaning they spend 4 times as many AP for movement, and can only move up to one-fourth their agility in addition to a static action in combat.
If they try not to be noticed while moving through a city, their speed should be treated as half their normal speed.\\
When moving stealthily, a creature makes a stealth check.
If that stealth check is below someone's~\nameref{stat:notice}, they realize something is off - they see moving shadows, hear footsteps, the hiding creature knocks something over etc.
The hiding creature might react to someone becoming aware of them, and the now aware creature might start looking around by making perception checks.

\subsection{Swimming}\label{subsec:swimming}
In still waters, a creature that can swim is usually considered to be able to swim normally.
Swimming for more than 1 hour without break will star to drain a creature's stamina at a rate of 1 per hour.\\
In rougher waters, a creature swimming has to make Agility checks to make sure they can overcome the currents.\\

\subsubsection{Suffocation}\label{subsubsec:suffocation}
A creature starts suffocating the moment they lose their ability to breathe.
A suffocating creature is effectively mute, and loses 1 Stamina per AP of suffocation.
After a suffocating creature has reached 0 Stamina, they start to take 1d6 damage every 5 AP.

\section{Carrying Capacity}\label{sec:carryingCapacity}
Items in the RLP are measured relative in the sizes "coin sized" (CS), "tiny" (T), "small" (S), medium (M), large(L), extra large (XL) or extra extra large (XXL).\\
One tiny object is equal to 10 coin sized items.
One small item is equal to 10 tiny items, and so forth.\\
These units are a combination of weight and volume, and containers usually can hold a certain amount of objects.
However, some objects are more or less dense.
For example, while a Backpack might be able to carry the equivalent of one large item in both weight and volume, it might not be able to hold 10 gold bars -- the straps of the backpack would possibly break.\\
Therefore, GMs and Players should treat these units as larger or smaller depending on the situation if they're dealing with denser or lighter materials.\\
A creature can usually carry one large item per point of ST, but might need a fitting container, like a backpack, to do so.\\

A person is usually able to carry their ST stat in large items.\\

\section{Hazards}\label{sec:hazards}
\subsection{Temperature}\label{subsec:temperature}
In regular environment, characters don't have to fear issues from temperature.
However, very high and very low temperatures can affect a character's health, stamina and can even damage them.\\

In warm and chilly environments, a character's stamina recovery is slowed down to 1 point every 2 AP of resting.
This can be countered by wearing armor with the warming or cooling abilities, respectively.\\

In hot and cold environments, a character takes 1 hot or cold damage per minute.
This damage is applied every 2 minutes, and requires a minimum damage reduction of 2 to be countered.\\

In extremely hot or extremely cold environments, a character takes 1 hot or cold damage per AP.
This damage is applied every 5 AP, and requires a minimum damage reduction of 2 to be countered.\\

\subsection{Falling}\label{subsec:falling}
Falling for \passus{2} deals 1d8 blunt damage on impact.
This damage increases by 1d8 for every additional \passus{2} travelled, to a maximum of \passus{50}, at which terminal speed is reached.
A character who can react and move can reduce this falling damage by making an Agility check.
The original DV for this check is 20, and it reduces the falling damage by 1d8.
For every 5 points above 20 that the Agility check reaches, it reduces the falling damage by an additional 1d8.
For example, a character falling \passus{16} would take 8d8 blunt damage on impact.
If said character makes an Agility check with a result of 27, they would only take 6d8 blunt damage instead.

\subsection{Traps}\label{subsec:traps}
Traps are - usually deliberately placed - hazards that consist of a trigger and a -- usually harmful -- effect.\\
When setting a trap, the creator usually rolls a~\nameref{sec:trapHandling} check to set the trap, and a~\nameref{sec:stealth} Check to hide it.\\
When a Character is in the area of a trap, compare the Trap's Finding DV to their Notice.

If the character is actively searching for traps, they gain +5 on their Notice for this comparison.

If the Notice is higher than the DV, the character notices that something is off.

They usually see some sign of the trigger.\\
If the character then decides to actively search for this specific trap, they roll a~\ref{sec:examination} check contested by the Trap's Find DV.
If they find the trap, they can then attempt to disable it by making a~\nameref{sec:trapHandling} check contested by the Trap's Trap Handling DV.\\
If the disarm attempt succeeds, the trap is disarmed.

If it fails, the character currently misses the knowledge or expertise to disarm this trap.\\
If the disarm attempt fails by 5 or more, the trap is sprung, and the effect is triggered immediately, regardless of if any characters are in the effect's area.\\
Additionally, if a character skilled in Trap Handling beats the DV of a Trap by 5 or more, they might be able to gather some or even all the materials used in the creation of the trap.\\
If a character triggers a trap, they usually realize that they have triggered something - they feel a tripwire that is being taught and a click coming from the wall, or they realize that the tile they stepped on is giving away, or something similar.

They then have the chance to react, gaining an appropriate check against the traps' effect if they react correctly.\\

\section{Units}\label{sec:units}
These units are Laetharnian standard Imperial units and are known around the world, thanks to the aggressive expansion tactics that the Empire employs.\\
These units are used throughout the document with their conversions provided.\\


\textbf{Fin} - Unit of length, about 0.6 inches or 1.5 centimeters\\
Usually used to measure things below 2 passus, like human size or the length of a weapon.\\


\textbf{Passus} - Unit of length, about 5 feet or 1.5 meters long\\
Used to measure short walking distances, measuring the distances in combat as an example.\\


\textbf{Leg} - Unit of length, about 800 passus, 1200 meters, 3940 feet\\
Used to measure great distances, like between cities.
Four Leg can usually be walked in an hour.\\


\textbf{Legal} - Unit of area, one Leg * one Leg\\
Used to measure property.
Half or quart a Legal are typical units for owned land.\\


\textbf{Stone} - Unit of mass, about 6 kg, 6 litres of water, or 13 pounds, 200 liquid ounces\\
Used for weighing of cattle or people.\\


\textbf{Pugnus} - Unit of mass, 20 Pugnus make a Stone, 10 Pugnus a Half Stone, so about 300 g, 300 ml or 0.66 pounds, 10 liquid ounces\\
Used for smaller measurements, like drinks or food.

A person usually needs 5 Pugnus of water per day.

"A Beer" from a tavern is usually a Pugnus, sometimes a double-Pugnus is referred to as a  "Large One", with an "Extra" being 4 Pugnus.\\

