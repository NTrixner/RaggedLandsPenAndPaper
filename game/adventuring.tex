\chapter{Adventuring}
\section{Units}
These units are Laetharnian standard Imperial units and are known around the world, thanks to the aggressive expansion tactics that the Empire employs.\\


\textbf{Fin} - Unit of length, about 0.6 inches or 1.5 centimeters\\
Usually used to measure things below 2 passus, like human size or the length of a weapon.\\


\textbf{Passus} - Unit of length, about 5 feet or 1.5 meters long\\
Used to measure short walking distances, measuring the distances in combat as an example.\\


\textbf{Leg} - Unit of length, about 800 legal, 1200 meters,  3940 feet\\
Used to measure great distances, like between cities. Four Leg can usually be walked in an hour.\\


\textbf{Legal} - Unit of area, one Leg * one Leg\\
Used to measure property. Half or quart a Legal are typical units for owned land.\\


\textbf{Stone} - Unit of mass, about 6 kg, 6 liters or 13 pounds, 200 liquid ounces\\
Used for weighing of cattle or people.\\


\textbf{Pugnus} - Unit of mass, 20 Pugnus make a Stone, 10 Pugnus a Half Stone, so about 300 g, 300 ml or 0.66 pounds, 10 liquid ounces\\
Used for smaller measurements, like drinks or food. A person usually needs 5 Pugnus of water per day. "A Beer" from a tavern is usually a Pugnus, sometimes a double-Pugnus is referred to as a  "Large One", with an "Extra" being 4 Pugnus.\\

\section{Carrying Capacity}
A person is usually able to carry their STR stat in Stone in addition to their body weight.\\



\section{Time, Speed and Travel}
\subsection{Time Units}
The shortest time-unit in the system is an AP, or Action Point. One Minute is considered to have 30 AP, therefore an AP equals two seconds.\\

The relations between seconds, minutes, hours and days are equivalent to earth units, even though they are somewhat longer units overall, since Kusa, the planet of the Ragged lands, is larger than earth.\\

A week on Kusa has 10 days, which are usually numbered (first weekday, second weekday, etc.), and the last 3 days are considered to be weekend days - even though that doesn't have much effect on common folk, who have to work in the fields and care for their animals anyway.\\

Each Kusaen Month consists of three weeks, i.e. 30 days.\\

A year begins mid-winter and consists of 13 months and one additional week. During this week, also called "black week" or "dark week", the unique constellation of the planet's two moons cause a week-long eclipse, which shrouds the planet in complete darkness. In this time, areas outside of towns are filled with monsters that disappear once the sun rises again. This is also the point at which the new year starts.\\

\subsection{Short-Distance Travel}
During Combat or in situations where split-second decisions are important, a character is assumed to be able to tactically move 1 Passus per AP, or dash 2 Passus per AP and Stamina spent. If a character is trying to move stealthy, they are considered to be half as fast during combat or during a time-based encounter.\\

\subsection{Mid-Distance Travel}
While travelling in areas with a dense layout, like a town, city or a cavern, a character is considered to be able to move 50 passus per minute. If a character tries to move stealthy in a dungeon, they are considered to move at one-fourth of this speed. If they try not to be noticed while moving through a city, their speed should be treated as half their normal speed.\\

\subsection{Long-Distance Travel}
While travelling overland, a character is considered to be able to move 4 Leg per hour, which means that a well-traveled character should be able to move 36 Legal per 8-hour interval; Adventurers are usually considered to be well-traveled.\\

One travelling day is sectioned into three separate 8-hour intervals, two of which are usually spent travelling. Travelling more than 16 hours reduces a character's maximum Stamina by 1 for each additional hour they spend travelling until they take a rest, at which point one maximum Stamina is restored for each hour rested.\\

For each 8-hour interval, each character in the group can choose a side activity. These consist of:
\begin{itemize}
	\item Follow Tracks
	
	\item Ensuring that the group moves stealthy (reduces the group's speed by half)
	
	\item Lookout for ambushes
	
	\item Salvaging the area for items
	
\end{itemize}

Depending on the area of travel and the group's makeup, there could be more options. For example, if the group is moving by horse-drawn wagon, one member of the group has to lead the wagon, while others could use the wagon to rest, learn a perk or even craft items.