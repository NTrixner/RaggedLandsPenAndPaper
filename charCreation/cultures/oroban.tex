\section{Oroban Culture}\label{culture:oroban}
You were raised in Oroban culture.
Orobans live in a segregated society, with Fortress Orobans living inside underground Fortresses in mountains, and Hillside Orobans living in the hills surrounding said fortress mountains.
Fortress Orobans are typically warriors, miners, smiths, stoneworkers, artisans, and priests.
Hillside Orobans often work as woodcutters, farmers, brewers, hunters, charcoal burners, carpenters, barrel makers, and traders.
As a result of this split society, most other above-ground folk typically only meet Hillside Orobans.
One Fortress, along with its' Hillside settlements, is usually ruled by a single mountain king, a title that's voted on periodically between multiple Fortress Oroban families.
Hillside Settlements provide the Fortress with resources, and are governed by sheriffs, a position directly appointed by the mountain kings, but as a lifelong title.
Oroban society is focused on honest work and hardship.
Orobans hold artisans and hard workers in high regard, and are strongly committed to their families, to the point of ancestor whorship.
The founders of an Oroban family line are known as paragons.
Orobans mostly whorship Prak, but in day to day life, they pray to their paragons, similar to other cultures praying to saints.
The oldest, and most important paragon of Oroban culture, is Orob the Mighty.
Orob, according to legend, was the first Oroban to leave the deep caves after his people were awoken by Prak.
As a result of Orob's heroic life and death, the Orobans named themselves after him.