\chapter{Process}
Creating a character is a daunting task, but here are some basic steps to go through if you want an interesting character with a bit of background.


\begin{enumerate}
	\item Decide on a basic idea that fits with the rest of the group, the story and the role your character should play in the group.

	\item Decide on a race

	\item Determine your character’s basic values

	\item Determine your character’s zodiac sign and social rank

	\item Use your starting amount of Character Points to buy Perks, Spells and Equipment

	\item Determine your character’s starting pool values
\end{enumerate}


Let’s go through the steps one by one.



\section{Decide on a basic idea}

Every character fits a theme. Do you want a swashbuckling buccaneer, a raging barbarian, a wise old wizard, maybe a young female priest going on a journey to test her faith?

You may get inspired by your favorite film or book character, or you may conjure up something completely new. Of course, characters are far more complex than just a basic idea, but at this stage of the creation process, you should create a rather simple stereotype that you want to follow. Try to describe your character with two or three words, or a short sentence.

Also try to think of the rest of the group, and the overall theme of the adventure you’re participating in. Playing a choleric barbarian in a game of intrigue and mystery may sound fun, but it will probably get stale pretty fast. Also, a group of 4 empathic thieves will also be pretty boring. 


\section{Determine your attribute values}

Now that you have an idea and a race, let’s talk numbers. You will have to assign your 7 basic attributes now. There is different ways to determine these values, and the GM may chose for the whole group.


\subsection{Point Buy}

When creating a character, start from a base of 2 for every value. Then assign your racial attributes. After that, you can assign 30 points freely to any of the 7 attributes, with a maximum of 12. Be sure to raise at least all of them to a value that you can live with. With an intelligence of 2, you’re not much smarter than your average wombat.


\subsection{Random roll}

Alternatively, you can roll 3D4 for every value, either directly assigning the values or rolling first and assigning them afterwards, and adding the racial bonuses afterwards. 



\subsection{Fixed Values}

Another alternative is using fixed values. When using this method, you gain the following values to assign to the attributes before applying racial bonuses: 10, 9, 9, 7, 5, 4, 3.



\section{Race, Zodiac signs and Background}

In the Ragged Lands, there is a great amount of different humanoid species. Seeing as you’re probably playing a somewhat intelligent creature, you are part of one of these species. Choose one that fits your character idea from this list, and note down the bonuses and maluses it gives you.

Your character may also be born under a certain star that may give him an additional, maybe even unique perk. You can determine this sign by random roll or you may choose one. However, your Gamemaster may force you to roll if he so desires.

Also, your character wasn’t just born into a culture, he also carries a certain social standing in said culture. He may have been born a slave, he may become a serf because of debt or maybe he was even born into nobility. Either roll randomly to determine your social rank or just start as a commoner. Your GM may chose a social rank for you based on your character's background. In that case, it shouldn't have an effect on your stats.



\section{Buy Spells, Equipment and Perks}

Every character usually starts with a total amount of 2000 Character Points and 1000 Gold. You can use these Points to buy your character’s starting perks and equipment.\\
When a character learns a perk that requires a source, they also have to pay the typical cost for that source at character creation.\\
These are the things your character has learned and acquired before the adventure starts, so be sure that it fits in your character’s background story, his personality, social rank and cultural background.\\




\section{Determine your pool values}

Now that you have your starting attribute values, you need your Health, Stamina and Mana. Every character starts with at least one level of the "Health Pool", "Stamina Pool" and "Mana Pool" Perks, rolling a specific die and then adding the relevant attribute. For Health, that’s Vitality, for Stamina, it’s Strength and for Mana its Intellect. When you bought Perks that grant you a higher die, these apply before rolling the dice now.